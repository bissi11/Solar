%%%%%%%%%%%%%%%%%%%%%%%%%%%%%%%%%%%%%%%%%
% Plain Cover Letter
% LaTeX Template
% Version 1.0 (28/5/13)
%
% This template has been downloaded from:
% http://www.LaTeXTemplates.com
%
% Original author:
% Rensselaer Polytechnic Institute 
% http://www.rpi.edu/dept/arc/training/latex/resumes/
%
% License:
% CC BY-NC-SA 3.0 (http://creativecommons.org/licenses/by-nc-sa/3.0/)
%
%%%%%%%%%%%%%%%%%%%%%%%%%%%%%%%%%%%%%%%%%

%----------------------------------------------------------------------------------------
%	PACKAGES AND OTHER DOCUMENT CONFIGURATIONS
%----------------------------------------------------------------------------------------

\documentclass[11pt]{article} % Default font size of the document, change to 10pt to fit more text

\usepackage{newcent} % Default font is the New Century Schoolbook PostScript font 
%\usepackage{helvet} % Uncomment this (while commenting the above line) to use the Helvetica font
\usepackage{hyperref}
\usepackage[square,sort,comma,numbers]{natbib}
\bibliographystyle{plain}
\setlength{\parskip}{\baselineskip}%
\setlength\parindent{0pt}
\usepackage{titlesec}


%\usepackage{letterbib}

% Margins
\topmargin=-1in % Moves the top of the document 1 inch above the default
\textheight=10.5in % Total height of the text on the page before text goes on 
%to the next page, this can be increased in a longer letter
\oddsidemargin=-10pt % Position of the left margin, can be negative or positive if you want more or less room
\textwidth=6.5in % Total width of the text, increase this if the left margin was decreased and vice-versa

\let\raggedleft\raggedright % Pushes the date (at the top) to the left, comment this line to have the date on the right

\begin{document}

%----------------------------------------------------------------------------------------
%	ADDRESSEE SECTION
%----------------------------------------------------------------------------------------

%\begin{letter} 

%----------------------------------------------------------------------------------------
%	YOUR NAME & ADDRESS SECTION
%----------------------------------------------------------------------------------------


%\signature{Bismark Singh} % Your name for the signature at the bottom

%----------------------------------------------------------------------------------------
%	LETTER CONTENT SECTION
%----------------------------------------------------------------------------------------

\today
\par
Dear Editor,
\par
We are re-submitting the manuscript, ``A Guide to Solar Power Forecasting using 
ARMA models'' for publication as a Brief Note in \textit{Solar Energy}.
\par
We have tried to incorporate almost all of the suggestions by both the 
Subject Editor and the Associate Editor. Specifically, we now state that our 
manuscript provides a guide for forecasting using ARMA models; we provide a 
comparison against both the smart persistence model and a single ARMA model (as 
opposed to hourly); we include a short introduction to ARMA models; and, we have 
expanded the list of references. We provide more details in the comment- 
to-the-editor file that we also provide.
\par
Various economical, social, and political factors support the increasing share 
of solar energy to complement energy generated using conventional fossil fuels 
in our electric power systems~\cite{stoddard2006economic}. Efficient and 
fast methods to forecast solar power are often needed for stochastic 
optimization models~\cite{su2014stochastic}, and easy-to-incorporate 
forecasting models are required.
\par
We provide a succinct step-by-step methodology methodology to build hourly 
solar power forecasts using historical data with an ARMA model. We provide 
conditions under which ARMA models are suitable; describe how to use 
various statistical tests and their suitability; and, provide metrics 
to estimate the fit. Finally, we use the model to develop future 
forecasts.

We believe the Brief Note will be useful for academicians and practitioners who 
are interested in solar power forecasting, as it provides directly 
implementable instructions to generate solar power scenarios. 
\par
Our manuscript is not under consideration for publication at any other journal. 
We look forward to hearing from you.
\par
Sincerely,
\newline
Bismark Singh
\newline
\textit{Sandia National Laboratories, Albuquerque, USA}
\newline
\noindent
David Pozo
\newline
\textit{Skolkovo Institute of Science and Technology, Moscow, Russia }

\par


For correspondence, use:
\newline
Bismark Singh, PhD
\newline
Discrete Math \& Optimization, Sandia National Laboratories, 
\newline
P.O. Box 5800, MS 1326
\newline
Albuquerque, NM, 87185-1320, USA
\newline
\href{mailto:bsingh@singh.gov}{bsingh@singh.gov}
%----------------------------------------------------------------------------------------
\vspace{-0.2in}
%\end{letter}

\begingroup
\titleformat*{\section}{\bf}
{\small
\bibliographystyle{IEEEtran}
\bibliography{mybibfile}
}
\endgroup
\end{document}