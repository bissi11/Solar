	%%%%%%%%%%%%%%%%%%%%%%%%%%%%%%%%%%%%%%%%%%%%%%%%%%%%%%%%%%%%%%%%%%%%%%%%%%%%
	%% Author template for Operations Reseacrh (opre) for articles with no e-companion (EC)
	%% Mirko Janc, Ph.D., INFORMS, mirko.janc@informs.org
	%% ver. 0.95, December 2010
	%%%%%%%%%%%%%%%%%%%%%%%%%%%%%%%%%%%%%%%%%%%%%%%%%%%%%%%%%%%%%%%%%%%%%%%%%%%%
	%\documentclass[opre,blindrev]{informs3}
	\documentclass[opre,nonblindrev]{informs3} % current default for manuscript submission
	
	\OneAndAHalfSpacedXI % current default line spacing
	%%\OneAndAHalfSpacedXII
	%%\DoubleSpacedXII
	%%\DoubleSpacedXI
	
	% If hyperref is used, dvi-to-ps driver of choice must be declared as
	%   an additional option to the \documentclass. For example
	%\documentclass[dvips,opre]{informs3}      % if dvips is used
	%\documentclass[dvipsone,opre]{informs3}   % if dvipsone is used, etc.
	
	%%% OPRE uses endnotes. If you do not use them, put a percent sign before
	%%% the \theendnotes command. This template does show how to use them.
	%\usepackage{endnotes}
	\let\footnote=\endnote
	\let\enotesize=\normalsize
	\def\notesname{Endnotes}%
	\def\makeenmark{$^{\theenmark}$}
	\def\enoteformat{\rightskip0pt\leftskip0pt\parindent=1.75em
	  \leavevmode\llap{\theenmark.\enskip}}
	% Private macros here (check that there is no clash with the style)
	\allowdisplaybreaks 
	\usepackage{graphicx}
	\usepackage{epstopdf}
	%\usepackage{subcaption}
	\usepackage{subfig}
	\usepackage{floatrow}
	%\floatsetup[subfigure]{capposition=top}
	\usepackage{multirow}
	\usepackage{amsmath}
	\usepackage{mathtools} 
	\usepackage{booktabs}
	
	\newcommand{\st}{\mbox{s.t.}}
	\newcommand{\E}{\mathbb{E}}
	\newcommand{\PP}{\mathbb{P}}
	
	
	% Natbib setup for author-year style
	\usepackage{natbib}
	\bibpunct[, ]{(}{)}{,}{a}{}{,}%
	\def\bibfont{\small}%
	\def\bibsep{\smallskipamount}%
	\def\bibhang{24pt}%
	\def\newblock{\ }%
	\def\BIBand{and}%
	
	\usepackage{url}
	% For nomenclature	
	\usepackage{nomencl}
	\makeglossary
	
	%% Setup of theorem styles. Outcomment only one.
	%% Preferred default is the first option.
	\TheoremsNumberedThrough     % Preferred (Theorem 1, Lemma 1, Theorem 2)
	%\TheoremsNumberedByChapter  % (Theorem 1.1, Lema 1.1, Theorem 1.2)
	\ECRepeatTheorems
	
	%% Setup of the equation numbering system. Outcomment only one.
	%% Preferred default is the first option.
	%\equationsNumberedThrough    % Default: (1), (2), ...
	%\equationsNumberedBySection % (1.1), (1.2), ...
	
	% In the reviewing and copyediting stage enter the manuscript number.
	%\MANUSCRIPTNO{} % When the article is logged in and DOI assigned to it,
			%   this manuscript number is no longer necessary
	
	%%%%%%%%%%%%%%%%
	\begin{document}
	%%%%%%%%%%%%%%%%
	
	% Outcomment only when entries are known. Otherwise leave as is and
	%   default values will be used.
	%\setcounter{page}{1}
	%\VOLUME{00}%
	%\NO{0}%
	%\MONTH{Xxxxx}% (month or a similar seasonal id)
	%\YEAR{0000}% e.g., 2005
	%\FIRSTPAGE{000}%
	%\LASTPAGE{000}%
	%\SHORTYEAR{00}% shortened year (two-digit)
	%\ISSUE{0000} %
	%\LONGFIRSTPAGE{0001} %
	%\DOI{10.1287/xxxx.0000.0000}%
	
	% Author's names for the running heads
	% Sample depending on the number of authors;
	% \RUNAUTHOR{Jones}
	% \RUNAUTHOR{Jones and Wilson}
	% \RUNAUTHOR{Jones, Miller, and Wilson}
	% \RUNAUTHOR{Jones et al.} % for four or more authors
	% Enter authors following the given pattern:
	\RUNAUTHOR{Singh et al.}
	
	% Title or shortened title suitable for running heads. Sample:
	% \RUNTITLE{Bundling Information Goods of Decreasing Value}
	% Enter the (shortened) title:
	\RUNTITLE{An optimal coordination of  solar photo-voltaic farm with bulk storage}
	
	% Full title. Sample:
	% \TITLE{Bundling Information Goods of Decreasing Value}
	% Enter the full title:
	\TITLE{A Proportionally Fair Criteria for Allocating Multiple Resources}
	
	% Block of authors and their affiliations starts here:
	% NOTE: Authors with same affiliation, if the order of authors allows,
	%   should be entered in ONE field, separated by a comma.
	%   \EMAIL field can be repeated if more than one author
	\ARTICLEAUTHORS{%
	\AUTHOR{Bismark Singh}
	\AFF{Discrete Math \& Optimization, Sandia National Laboratories, Albuquerque, NM 87185 \\
	\EMAIL{bsingh@sandia.gov}}
	 
	\AUTHOR{David Pozo}
	\AFF{Skoltech Center for Energy Systems, Skolkovo Institute of Science and Technology, Moscow, Russia \\ \EMAIL{d.pozo@skoltech.ru}} %, \URL{}}
	

	} % end of the block
	
	\ABSTRACT{%
	}%
	
	% Sample
	%\KEYWORDS{deterministic inventory theory; infinite linear programming duality;
	%  existence of optimal policies; semi-Markov decision process; cyclic schedule}
	
	% Fill in data. If unknown, outcomment the field
	\KEYWORDS{proportional fairness; equity; optimization; resource allocation; healthcare} \HISTORY{%This paper was first submitted on April 12, 1922 and has been with the authors for 83 years for 65 revisions.
	}
	\maketitle
	%%%%%%%%%%%%%%%%%%%%%%%%%%%%%%%%%%%%%%%%%%%%%%%%%%%%%%%%%%%%%%%%%%%%%%
	
	% Samples of sectioning (and labeling) in OPRE
	% NOTE: (1) \section and \subsection do NOT end with a period
	%       (2) \subsubsection and lower need end punctuation
	%       (3) capitalization is as shown (title style).
	%
	%\section{Introduction.}\label{intro} %%1.
	%\subsection{Duality and the Classical EOQ Problem.}\label{class-EOQ} %% 1.1.
	%\subsection{Outline.}\label{outline1} %% 1.2.
	%\subsubsection{Cyclic Schedules for the General Deterministic SMDP.}
	%  \label{cyclic-schedules} %% 1.2.1
	%\section{Problem Description.}\label{problemdescription} %% 2.
	
	% Text of your paper here
	\section{Notation}

			\begin{tabular}{l l}
				
				\bf {Sets and Indices} & \\
				$t \in T$ 		 & set of hours\\
				$w \in \Omega$ 		& set of scenarios \\

				\bf {Data and Parameters} & \\
				$\eta_c$ 		& battery's charging efficiency, $ 0 < \eta_c < 1 $   \\
				$\eta_d$ 		& battery's discharging efficiency, $ 0 < \eta_d < 1$   \\
				$s_t^\omega$ & solar energy available at hour $t$ under scenario $\omega$ [MWh]   \\ 
				$\underline{X}, \overline{X}$ 		& upper and lower bounds on battery's capacity [MWh]  \\
				$\underline{Y}, \overline{Y}$ 		& upper and lower bounds on battery's charging per hour  [MWh]  \\
				$\underline{Z}, \overline{Z}$ 		& upper and lower bounds on batter's discharging per hour [MWh]  \\\\
				
				\bf {Decision Variables} & \\
				$r_t$ 		     		   & energy promised to deliver at hour $t$ [MWh]  \\
				$x_t^\omega$ 		 & energy stored in battery at hour $t$ under scenario $\omega$ [MWh]  \\
				$y_t^\omega$ 		 & energy charged to battery at hour $t$ under scenario $\omega$ [MWh]  \\
				$z_t^\omega$ 		 & energy discharged to battery at hour $t$ under scenario $\omega$ [MWh]  \\
			\end{tabular}
		\label{tab:notation}
	
	\section{Introduction}
	
	We consider a problem where a solar photovoltaic power station (SPP) commits (e.g., on power auctions) delivering $r_t$ MWh of energy at the hour $t$. Solar production is considered uncertain and represented with the random parameter $\tilde{s}_t$. The solar production is zero during some hours a day.
	Within the SPP, there is a bulk storage, e.g., pump hydro generator or Li-ion batteries, with a given storage capacity $ \overline{X}$. The bulk storage is used to meet the uncertainty of the solar generation in order to meet the contract commitments. Also, it can be used for selling/purchase energy in the pool-based market at price $\lambda_t^q$/$\lambda_t^p$ ($\lambda_t^p \ge \lambda_t^q$). The purchased/sold energy, $p_t$/$q_t$, from/to the pool-based market has to be decided a day-ahead of the energy delivery. Observe, that energy purchased $p_t$ from the pool-based market would make  possible to met the contract for days with very little solar generation. 
	
	The objective consist of maximizing profits from selling the committed $r_t$ MWh and the bulk storage considering the joint coordination of the SPP and bulk storage.  We do not consider ({\color{red} for now.  We may can include it in the future, it shouldn't change the solution methodology}) penalization on not meeting the committed energy. 
	
    State-of-charge (SoC) of the storage has be withing it safety limits.
	

	
	The following optimization model summarizes the above described problem. 

	{
	\begin{subequations} \label{eqn: solar_1}
	\begin{eqnarray}
		\min_{x,y,z,p,q} & \sum_{t \in t} (\lambda^p_t p_t -  \lambda_t^q q_t)     & \label{eqn: solar_obj} \\
		\mbox{s.t.} &    \PP \left\{ r_t  \geq s_t^{\omega} + z_t^\omega -  y_t^\omega,  \forall t\in T  \right\} \ge 1-\alpha & \label{eqn. commit} \\
		& x_{t+1}^\omega = x_t^\omega +  \eta_c (y_t^\omega + p_t) - \frac{1}{\eta_d} (z_t^\omega + q_t),  & \forall t \le |T|-1        \label{eqn: solar_2}\\
		& 	 \underline{X} \le x_t^\omega \le \overline{X}, & \forall t \in T 	\label{eqn: solar_3}\\
		& 	 (y_{t+1}^\omega + p_{t+1}) - (y_{t}^\omega + p_{t})  \le R^{up},  &  \forall t \le |T|-1   \label{eqn: solar_4}\\
		& 	 (z_{t}^\omega + q_{t}) - (z_{t+1}^\omega + q_{t+1})  \le R^{dw},  &  \forall t \le |T|-1   \label{eqn: solar_5} \\
		 & y_{t}^\omega, z_{t}^\omega , p_t, q_t  \ge 0 & \label{eqn: solar_6} 
	\end{eqnarray}
	\end{subequations}
	}

The objective function aims to minimize purchased energy in the day-ahead market and maximize the energy sold to it. Equation \eqref{eqn. commit} enforce that energy delivered by the contract has to come from the solar production or storage. Equation \eqref{eqn: solar_2} represents the SoC of the storage. Note that $y_t^\omega$ represents the energy sent to the storage from the SPP excess production while $p_t$ is the energy sent to the storage that had been bought to the pool-based market. Equation \eqref{eqn: solar_3} represents the technical limits of the storage.  Equations \eqref{eqn: solar_5} and \eqref{eqn: solar_6} represents maximum charging and discharging rate of the storage device. 





	%   (while the paper is being written).
	\bibliographystyle{ormsv080} % outcomment this and next line in Case 1
	\bibliography{thebib2} % if more than one, comma separated
	
	% CASE 2: BiBTeX used to generate mypaper.bbl (to be further fine tuned)
	%\input{mypaper.bbl} % outcomment this line in Case 2
	
	%If you don't use BiBTex, you can manually itemize references as shown below.
	
	%%%%%%%%%%%%%%%%%
	\end{document}
	%%%%%%%%%%%%%%%%%
